% The Flask thing at UDS
\documentclass[xcolor=dvipsnames]{beamer}
\usepackage{graphicx}
\usepackage{caption}
\usepackage{verbatim}
\def\us{\char`\_}
\usetheme{CambridgeUS}

\title{Deploying with Fabric}
\author{Nigel}
\date[phpcloud]{Scaling PHP in the cloud}

\begin{document}

    \begin{frame}
        \titlepage
    \end{frame}

    \begin{frame}{What the heck is Fabric?}
        \begin{itemize}
            \pause \item Well, for starters, its written in python.  \pause Yes, I'm talking about Python in a PHP conference.
            \pause \item Its a library and a command-line tool.
            \pause \item It streamlines the use of SSH for application deployment or systems administration tasks.
            \pause \item That doesn't mean you have to use it only for SSH related tasks.
            \pause \item You know those boring and repetitive tasks you have to do for every deployment?  Writing it in fabric removes the monotony, it'll be one step!
            \pause \item Fabric can execute local or remote commands and upload/download files and much much more!
        \end{itemize}
    \end{frame}

    \begin{frame}{Ok, so how do I use Fabric?}
        \begin{center}
        \pause Write a file called fabfile.py, write your commands in it, and place it in the root of your project.  Then execute individual commands with "fab foo".
        \end{center}
    \end{frame}
    
    \begin{frame}{This is what a fabfile.py looks like}
        \begin{verbatim}

            from fabric.api import run \\

            def host\us type():\\
            \hspace{20 pt}run('uname -s')\\
            
        \end{verbatim}

    \end{frame}


    \begin{frame}{What happens when you run it}
        \begin{verbatim}

            \$ fab -H localhost,linuxbox host\us type \\
            $[$localhost$]$ run: uname -s \\
            $[$localhost$]$ out: Darwin \\
            $[$linuxbox$]$ run: uname -s \\
            $[$linuxbox$]$ out: Linux \\

            Done. \\
            Disconnecting from localhost... done. \\
            Disconnecting from linuxbox... done. \\
            
        \end{verbatim}

    \end{frame}

    \begin{frame}{What makes Fabric unique?}
        \begin{center}
        \pause There are other deployment tools out there, what makes Fabric unique?  Fabric gives you a basic API on which YOU build the methods for YOUR workflow.
        \end{center}
    \end{frame}

    \begin{frame}{Lets look at the API further}
        \begin{center}
        \pause There's no point in me saying Fabric is the next best thing since sliced bread without showing off its API.
        \end{center}
    \end{frame}

    \begin{frame}{Local commands}
        \begin{center}
        \pause There's no point in me saying Fabric is the next best thing since sliced bread without showing off its API.
        \end{center}
    \end{frame}
    
    \begin{frame}{Local commands}
        \begin{verbatim}
            from fabric.api import local\\

            def prepare\us deploy():\\
            \hspace{20 pt}local("bin/run tests")\\
            \hspace{20 pt}local("git add . \&\& git commit")
            
        \end{verbatim}

    \end{frame}  

    \begin{frame}{Lets go remote}
        \begin{verbatim}

            from \us\us future\us\us import with\us statement \\
            from fabric.api import local, settings, abort, run, cd \\
            from fabric.contrib.console import confirm \\

            def deploy(): \\
            \hspace{15 pt}code\us dir = '/var/www/myproject' \\
            \hspace{15 pt}with settings(warn\us only=True): \\
            \hspace{15 pt}\hspace{15 pt}if run("test -d \%s" \% code\us dir).failed: \\
            \hspace{15 pt}\hspace{15 pt}\hspace{15 pt}run("git clone user@github.com/user/repo" \% code\us dir) \\
            \hspace{15 pt}with cd(code\us dir): \\
            \hspace{15 pt}\hspace{15 pt}run("git pull") \\
            
        \end{verbatim}
    \end{frame}

    \begin{frame}{What happens when you run it}
        \begin{verbatim}
    
            \$ fab deploy \\
            No hosts found. Please specify (single) host \\string for connection: myserver \\
            $[$myserver$]$ run: git pull \\
            $[$myserver$]$ out: Already up-to-date. \\
            $[$myserver$]$ out: \\
            $[$myserver$]$ run: touch app.wsgi \\

            Done. \\

        \end{verbatim}
    \end{frame}

    \begin{frame}{Use the env}
        \begin{verbatim}
    
            env.hosts = ['my\us server']

        \end{verbatim}
    \end{frame}    

    \begin{frame}{Disadvantage}
        \begin{itemize}
            \pause \item Not parallel.  If you have to execute tasks on 5 machines, Fabric does it one by one.  \pause However, there is a fork of Fabric that can execute tasks parallely on multiple hosts.
            \pause \item Everything needs to be defined manually.  Yeah, this is the power as well as sometimes the curse.  For certain situations, its just easier to use tools like capistrano.
        \end{itemize}
    \end{frame}    

    \begin{frame}{Workflow Discussions}
        \begin{center}
        \pause How do you deploy your sites to production?
        \end{center}
    \end{frame}  

    \begin{frame}{More information}
        \begin{itemize}
        \item Fabric Website, http://fabfile.org
        \item \LaTeX Source for my slides, https://github.com/nigelbabu/phpcloud-fabric
        \end{itemize}
    \end{frame}

    \begin{frame}{Q and A}
        \begin{center}
        \pause Ask your questions, give your feedback. \\
        \pause \small Complains redirected to /dev/null.
        \end{center}
    \end{frame}  

\end{document}
